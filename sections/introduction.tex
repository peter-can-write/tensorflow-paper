\section{Introduction}

Modern artificial intelligence systems and machine learning algorithms have
revolutionized approaches to scientific and technological challenges in a
variety of fields. We can observe remarkable improvements in the quality of
state-of-the-art computer vision, natural language processing, speech
recognition and other techniques. Moreover, the benefits of recent breakthroughs
have trickled down to the individual, improving everyday life in numerous
ways. Personalized digital assistants, recommendations on e-commerce platforms,
financial fraud detection, customized web search results and social network
feeds as well as novel discoveries in genomics have all been improved, if not
enabled, by current machine learning methods.

A particular branch of machine learning, \emph{deep learning}, has proven
especially effective in recent years. Deep learning is a family of
representation learning algorithms employing complex neural network
architectures with a high number of hidden layers, each composed of simple but
non-linear transformations to the input data. Given enough such transformation
modules, very complex functions may be modeled to solve classification,
regression, transcription and numerous other learning tasks \cite{nature2015}.

For real world usage, deep learning algorithms must eventually be transcribed
into a computer program. There exist a number of machine learning software
libraries and frameworks for this purpose. Among these are Theano \cite{theano},
Torch \cite{torch}, scikit-learn \cite{scikit} and many more. In November 2015,
this list was extended by \emph{TensorFlow}, a novel machine learning software
library released by Google \cite{tensorflow}. As per the initial publication,
TensorFlow aims to be ``an interface for expressing machine learning
algorithms'' in ``large-scale [\dots] on heterogeneous distributed systems''
\cite{tensorflow}.

The remainder of this paper aims to give a thorough review of TensorFlow and is
further structured as follows. Section \ref{sec:model} discusses in depth the
computational paradigms underlying TensorFlow. In Section \ref{sec:code} we then
explain the current programming interface. Tools allowing to visualize models
built with TensorFlow are studied in Section \ref{sec:visual}. Subsequently,
Section \ref{sec:comp} compares TensorFlow to alternative deep learning
libraries on a quantitative basis. Before concluding our review in Section
\ref{sec:conclusion}, Section \ref{sec:uses} investigates current real world
uses of TensorFlow in literature and industry.

%%% Local Variables:
%%% mode: latex
%%% TeX-master: "../paper"
%%% End:
