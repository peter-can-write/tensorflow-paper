\begin{abstract}

Deep learning is a subfield of Artificial Intelligence employing deep neural
network architectures that has brought along breakthroughs in computer vision,
speech recognition, natural language understanding and other domains. In
November 2015, Google released \emph{TensorFlow}, an open source deep learning
software library enabling definition, training and deployment of machine
learning models. In this paper, we review TensorFlow and put in context of
state-of-the-art deep learning paradigms and software. We discuss its basic
computational paradigms and distributed execution model, its programming
interface as well as available model visualization toolkits. Next to
TensorFlow, there exist other deep learning frameworks such as Theano or
Torch. We compare TensorFlow to these alternatives on a qualitative and
quantitative basis and finally comment on observed real-world uses of
TensorFlow today.

\end{abstract}

%%% Local Variables:
%%% mode: latex
%%% TeX-master: "../paper"
%%% End:
